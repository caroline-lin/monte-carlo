\documentclass{beamer}
\usepackage{amsmath,amssymb,amsthm,bbm,bm, amsfonts}
\usepackage{algorithm}
\usepackage{algpseudocode}
\setbeamertemplate{footline}[frame number]
\begin{document}
% Slide Outline:
% Laplace's Problem
% Poisson's Problem (A Generalization)
% Possible Approaches (Finite Whatever)
\begin{frame}{Finite Differences}
One common approach: solve using the 2D discrete Laplacian.
	\[ L = \frac{1}{h^2} \begin{bmatrix} 
		0 & 1 & 0 \
		1 & -4& 1 \
		0 & 1 & 0 
	\end{bmatrix} \]
Pros: solving becomes a matrix solve
Boundary conditions are encoded into a vector
Cons: Needs a Cartesian grid

Alternatives: Finite elements, finite volumes.
\end{frame}
% Random Walks
\begin{frame}{Random Walks in a nutshell}
A "normally distributed" stochastic process in time, i.e.
	
$E(B(t)) \sim N(0, t)$ for all times $t \geq 0$
\end{frame}

\begin{frame}{Laplace's Equation with Random Walks}
Ito's formula:
$u(x) = E^x[ g(\mathbf{B}(T))] + \frac{1}{2} E^x \left[ \int_0^T f(B(s))ds \right]$ {#eq:rw}

For Laplace's, take the mean over the boundary values, weighted by frequency of random walks.
\end{frame}
% Why It Works (In Theory)

% Extensions: Interior Finite Differences

% Extensions: Self-Contacting Boundary Coupling

% Extensions: Walk On Spheres
% Results: Accuracy
% Results: Spiky Domains
% Results: Runtime
% Conclusions
\begin{frame}{Extension: Interior Finite Differences}
However, using only random walks can be very slow and inefficient.
Having to do many random walks for each point we can about can be very slow.
\end{frame}
\end{document}
