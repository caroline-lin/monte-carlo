\documentclass{beamer}
\newcommand\di{\partial}
\newcommand\pars[1]{\left(#1\right)}
\newcommand\mat[1]{#1}
\usepackage{amsmath,amssymb,amsthm,bbm,bm, amsfonts}
\usepackage{algorithm}
\usepackage{algpseudocode}
\setbeamertemplate{footline}[frame number]
\title{Solving Laplace's Problem Via Random Walks}
\author{Caroline Lin and Dmitry Paramonov}
\date{December 4th, 2018}

\begin{document}
\begin{frame}
  \titlepage%
\end{frame}
% Slide Outline:
% Laplace's Problem
% Poisson's Problem (A Generalization)
\begin{frame}{Laplace's Problem}
\begin{itemize}
\item We wish to solve \emph{Laplace's Problem}.
Here, we have some closed domain $\Omega$ with a boundary $\di\Omega$.
We are given a function $g\pars{\vec{x}}$, defined on $\vec{x}\in\di\Omega$.
\item We wish to find a function $u\pars{\vec{x}}$, defined on $\Omega$,
such that $\Delta u=0$ on $\Omega$
and such that $u\pars{\vec{x}}=g\pars{\vec{x}}$ on $\di\Omega$.
\item A generalization to this is \emph{Poisson's Problem},
wherein we also have some $f\pars{\vec{x}}$ defined on $\vec{x}\in\Omega$,
and where we require that $\Delta u\pars{\vec{x}}=f\pars{\vec{x}}$ on $\Omega$.
\end{itemize}
\end{frame}

% Possible Approaches (Finite Whatever)
\begin{frame}{Finite Differences}
One possible approach: solve using the 2D discrete Laplacian.
	\[ L = \frac{1}{h^2} \begin{bmatrix}
		0 & 1 & 0 \\
		1 & -4& 1 \\
		0 & 1 & 0
	\end{bmatrix} \]
\begin{itemize}
\item Pros: solving becomes a matrix solve
\item Pros: Boundary conditions are encoded into a vector
\item Cons: Needs a Cartesian grid
\end{itemize}

Alternatives: Finite elements, finite volumes.
\end{frame}
% Random Walks
% Why It Works (In Theory)
\begin{frame}{Random Walks in a nutshell}
A ``normally distributed'' stochastic process in time, i.e.

$E(B(t)) \sim N(0, t)$ for all times $t \geq 0$

Ito's formula:
$u(x) = E^x[ g(\mathbf{B}(T))] + \frac{1}{2} E^x \left[ \int_0^T f(B(s))ds \right]$

For Laplace's, take the mean over the boundary values, weighted by frequency of random walks.
Why it works heuristically: Laplace's equation models a diffusion process. 
\end{frame}

% Extensions: Interior Finite Differences
\begin{frame}{Extension: Interior Finite Differences}
\begin{itemize}
\item However, random walks are slow.
% \item However, using only random walks can be very slow and inefficient.
% Having to do many random walks for each point we can about can be very slow.
\item Instead, define a Cartesian grid within the main boundary.
\item Can use finite differences on the inside of this grid.
\item The boundary values can be found using random walks.
\item Amount of random walks now scales with $h^{-1}$, instead of $h^{-2}$.
\end{itemize}
\end{frame}

% Extensions: Self-Contacting Boundary Coupling
\begin{frame}{Extensions: Self-Contacting Boundary Coupling}
\begin{itemize}
\item Furthermore, we can add coupling to the grid points.
\item If a random walk touches the interior boundary,
then the expected value of that random walk
is the expected value of a random walk starting at this new point.
\item We thus store an extra matrix $\mat{F}$.
$F_{i,j}$ is the number of random walks starting at boundary point $i$
and ending at boundary point $j$.
If $K$ is the number of random walks performed
and $\vec{b}$ is the vector of total values obtain by random walks,
we solve
$$\vec{u}=\frac{1}{K}\mat{F}\vec{u}+\frac{1}{K}\vec{b},$$
\end{itemize}
\end{frame}

% Extensions: Walk On Spheres
\begin{frame}{Extensions: Walk On Spheres}
\begin{itemize}
	\item Applicable for Laplace's equation: Select a ball contained in $D$ 
	\item Time-step by randomly selecting a point on the sphere (saves time)
\end{itemize}
\end{frame}
% Results: Accuracy

\begin{frame}
	\begin{centering}
	\includegraphics[width=0.8\textwidth]{circle_04.pdf}
	\end{centering}
\end{frame}
\begin{frame}
	\begin{centering}
	\includegraphics[width=0.8\textwidth]{circle_02.pdf}
	\end{centering}
\end{frame}
\begin{frame}
	\begin{centering}
	\includegraphics[width=0.8\textwidth]{circle_01.pdf}
	\end{centering}
\end{frame}

\begin{frame}
Results for the circle, no coupling
N is the number of points in each dimension.

\begin{center}
\begin{tabular}{|c|c|c|c|c|}
	\hline
	N & K & Max abs err & Max rel err & Mean rel err \\  
	\hline 
	50 & 25 & 0.1167 & 21.6 & 0.033474 \\
	50 & 100 & 0.07035 & 8.1546 & 0.02415 \\
	50 & 400 & 0.03810 & 20.683& 0.0302 \\
	100 & 25 & 0.1176 & 36.625 & 0.01340 \\
	100 & 100 & 0.06578 & 7.22489 & 0.00755 \\
	100 & 400 & 0.01781 & 5.2412 & 0.002187 \\
	50 & "$\inf$" & 1.22e-14 & 2.5057e-12 & 9.2972e-15 \\
	100 & "$\inf$" & 5.2402e-14 & 1.1273e-11 & 3.3339e-14 \\
	\hline
\end{tabular}
\end{center}
\end{frame}

% Results: Spiky Domains

\begin{frame}
	\begin{centering}
	\includegraphics[width=0.8\textwidth]{squiggly_05.pdf}
	\end{centering}
\end{frame}
\begin{frame}
	\begin{centering}
	\includegraphics[width=0.8\textwidth]{squiggly_013.pdf}
	\end{centering}
\end{frame}

\begin{frame}
\begin{center}
	\begin{tabular}{|c|c|c|c|c|}
		\hline
		N & K & Max abs err & Max rel err & Mean rel err \\  
		\hline 
		40 & 25 & 0.1769 & 1.4410 & 0.0435 \\
		40 & 100 & 0.06414 & 0.5077 & 0.0225 \\
		50 & 400 & 0.04473 & 0.4936 & 0.00215 \\
		150 & 25 & 0.1156 & NA & NA \\
		150 & 100 & 0.0548 & NA & NA \\
		150 & 400 & 0.02929 & NA & NA \\
		40 & "$\inf$" & 1.77e-15 & 5.3062e-15 & 3.4504e-16 \\ 
		150 & "$\inf$" & 1.8318e-14 & NA & NA \\
		\hline
	\end{tabular}
	\end{center}
\end{frame}

% Results: Runtime

\begin{frame}
	\begin{centering}
		\begin{tabular}{|c|c|c|c|}
			\hline
			N&K&Simulation Time&Solve Time\\
			\hline
			50&25&3.11&0.69\\
			50&100&12.27&0.69\\
			50&400&49.25&0.71\\
			100&25&12.59&2.72\\
			100&100&48.42&2.69\\
			100&400&181.71&2.72\\
			\hline
			\end{tabular}
	\end{centering}
	
\end{frame}

% TODOs
\begin{frame}{Things we didn't have time for}
	\begin{itemize}
		\item Random-walk acceleration on the GPU
		\item Can we improve the coupling?
		\item Profiling Finite-Elements against Random Walk on ill-formed domain
	\end{itemize}
\end{frame}

% Conclusions
\end{document}
