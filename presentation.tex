\documentclass{beamer}
\usepackage{amsmath,amssymb,amsthm,bbm,bm, amsfonts}
\usepackage{algorithm}
\usepackage{algpseudocode}
\setbeamertemplate{footline}[frame number]
\begin{document}
% Slide Outline:
% Laplace's Problem
% Poisson's Problem (A Generalization)
% Possible Approaches (Finite Whatever)
% Random Walks
% Why It Works (In Theory)
% Extensions: Interior Finite Differences
% Extensions: Self-Contacting Boundary Coupling
% Extensions: Walk On Spheres
% Results: Accuracy
% Results: Spiky Domains
% Results: Runtime
% Conclusions
\begin{frame}{Extension: Interior Finite Differences}
However, using only random walks can be very slow and inefficient.
Having to do many random walks for each point we can about can be very slow.
\end{frame}
\end{document}
